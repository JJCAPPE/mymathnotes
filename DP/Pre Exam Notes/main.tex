\documentclass{report}

\input{preamble}
\input{macros}
\input{letterfonts}

\usepackage{tikz}
\usepackage{tikz-3dplot}
\usepackage{amsmath}
\usepackage{pgfplots}

\title{\Huge{Pre-Exam Notes}\\DP2}

\author{\huge{Giacomo Cappelletto}}
\date{29.4.24}

\begin{document}

\maketitle
\newpage
\tableofcontents
\chapter{To Remember}

\section{Functions}

\dfn{Column Transfromation of a graph}{
	\textbf{If the transformation for $f(x)$ is }
	$$\begin{pmatrix}
		3 \\
		5
	\end{pmatrix}$$
	\textbf{Then the new function is }
	$$f(x-3)+5$$
}

\section{Series}

\dfn{Induction Logic}{
	\centering
		\textbf{Since true for $n=1$ and ture for $n=k$ implies true for $n=k+1$, true for all $n \in \mathbb{Z}^+$}
	
}

\section{Probability}

\nt{
	\centering
		\textbf{Always remember to include swapping in combination problems}
	}

\section{Trig and Geometry}

\dfn{Tangent Table}{
	\begin{center}
		\begin{tabular}[pos]{c | c}
			Angle ($\deg $) & $tan(\theta)$        \\
			\hline
			0               & 0                    \\
			30              & $\frac{1}{\sqrt{3}}$ \\
			45              & $1$                  \\
			60              & $\sqrt{3}$           \\
			90              & undef
		\end{tabular}
	\end{center}
}

\dfn{Scalar Product Angle in Parallelograms}{
	If $ cos(\theta) < 0$ then $\theta$ is obtuse
	The $\theta$ found is always the one with the matching edge
}
\ex{$cos(\theta)$ Edge}{
	\begin{center}
		\begin{tikzpicture}
			% Define the coordinates of the parallelogram
			\coordinate (A) at (0,0);
			\coordinate (B) at (3,0);
			\coordinate (C) at (4,2);
			\coordinate (D) at (1,2);

			% Draw the parallelogram
			\draw (A) -- (B) -- (C) -- (D) -- cycle;

			% Label the vertices with nodes
			\node at (A) [below left] {$A$};
			\node at (B) [below right] {$B$};
			\node at (C) [above right] {$C$};
			\node at (D) [above left] {$D$};

		\end{tikzpicture}
	\end{center}

	$$\vec{AB} \cdot \vec{AD} = cos\left\langle B\mathbf{A} D \right\rangle \left\lvert \vec{AB} \right\rvert \left\lvert \vec{AD} \right\rvert$$
}

\section{Calculus}

\qs{Integral of a known derivative}{
	$$\int_{}^{} f(x) \,dx$$
	Where $f'(x)$ is known
}

\sol Integrate by parts as 
\boldmath $$ du = f(x) $$ \boldmath $$ v = 1 $$

\nt{You can flip a derivative if needed}
\ex{Given $\frac{dy}{dx} = \frac{5 cos(x)}{4}$}{
	$$\frac{dx}{dy} = \frac{4}{5 cos(x)}$$
}

\end{document}