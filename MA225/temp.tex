\documentclass{report}

\input{preamble}
\input{macros}
\input{letterfonts}

\usepackage{tikz}
\usepackage{tikz-3dplot}
\usepackage{amsmath}
\usepackage{amssymb}
\usepackage{pgfplots}
\usepgfplotslibrary{polar}
\pgfplotsset{compat=newest}
\usepackage{smartdiagram}
\usepackage{xcolor}
\usepackage{forest}
\usepgfplotslibrary{colormaps}
\usepgfplotslibrary{groupplots}

%added
\tikzset{>=latex}
\usepackage{siunitx}
\usepackage[outline]{contour} % glow around text
\usetikzlibrary{angles,quotes} % for pic
\contourlength{1.3pt}
\usetikzlibrary{math}
\usepackage{ifthen}
\tikzset{>=latex} % for LaTeX arrow head
\usepackage{xcolor}
\colorlet{veccol}{green!70!black}
\colorlet{vcol}{green!70!black}
\colorlet{xcol}{blue!85!black}
\colorlet{projcol}{xcol!60}
\colorlet{unitcol}{xcol!60!black!85}
\colorlet{unitcol2}{vcol!60!black!85}
\colorlet{myblue}{blue!70!black}
\colorlet{myred}{red!70!black}
\tikzstyle{vector}=[->,very thick,xcol]
\tikzstyle{mydashed}=[dash pattern=on 2pt off 2pt]
\def\tick#1#2{\draw[thick] (#1) ++ (#2:0.1) --++ (#2-180:0.2)} %0.03*\xmax


\usesmartdiagramlibrary{additions}

\title{\Huge{Temporary Doc}\\Calc 3}
\author{\huge{Giacomo Cappelletto}}
\date{23/10/24}

\begin{document}


\maketitle
\pagebreak
\pdfbookmark[section]{\contentsname}{toc}
\tableofcontents
\pagebreak


\chapter{Vector Valued Functions $f:\mathbb{R} \rightarrow \mathbb{R}^n$}

\section{Change of Variable for Double and Triple Integrals}

\subsection*{Polar Coordinates}
\[
	\iint_D f(x, y) \, dx \, dy \rightarrow \iint_S f(r\cos\theta, r\sin\theta) \, r \, dr \, d\theta
\]

\subsection*{Cylindrical Coordinates}
\[
	\iiint_D f(x, y, z) \, dx \, dy \, dz \rightarrow \iiint_S f(r\cos\theta, r\sin\theta, z) \, r \, dr \, d\theta \, dz
\]

\subsection*{Spherical Coordinates}
\[
	\iiint_D f(x, y, z) \, dx \, dy \, dz \rightarrow \iiint_S f(\rho\sin\phi\cos\theta, \rho\sin\phi\sin\theta, \rho\cos\phi) \, \rho^2 \sin\phi \, d\rho \, d\phi \, d\theta
\]

\thm{Intuition Behind Change of Variables}{
We use a \textbf{mapping} \( T \) to transform coordinates in one space \( S \) to another \( R \). This is particularly useful when integrating over regions that are easier to describe in new coordinates (e.g., circular or spherical regions).

For example:
\[
	S = [0, 2\pi] \times [0, 2], \quad T(r, \theta) = (r\cos\theta, r\sin\theta)
\]

Here, the mapping \( T \) converts a point in \( S \) into a point in \( R \).


\subsection*{Area Differential Transformation}
Consider a small differential area element in the original space:
\[
	dA = |\det(J)| \, du \, dv
\]
where \( J \) is the \textbf{Jacobian matrix}, and \( |\det(J)| \) accounts for how the transformation scales area.
}

\dfn{Jacobian Matrix}{
The Jacobian matrix represents the linear transformation of the mapping \( T \) at a given point:
\[
	J = \begin{bmatrix}
		\frac{\partial x}{\partial u} & \frac{\partial x}{\partial v} \\
		\frac{\partial y}{\partial u} & \frac{\partial y}{\partial v}
	\end{bmatrix}
\]

For a transformation \( T(u, v) = (g(u, v), h(u, v)) \), the determinant of \( J \) is:
\[
	\det(J) =
	\begin{vmatrix}
		\frac{\partial g}{\partial u} & \frac{\partial g}{\partial v} \\
		\frac{\partial h}{\partial u} & \frac{\partial h}{\partial v}
	\end{vmatrix}
	= \frac{\partial g}{\partial u} \cdot \frac{\partial h}{\partial v} - \frac{\partial g}{\partial v} \cdot \frac{\partial h}{\partial u}
\]

}

\subsection*{Geometric Interpretation}
\begin{itemize}
	\item \textbf{Local Stretching/Scaling:} \( |\det(J)| \) gives the local scaling factor of the area due to the transformation.
	\item \textbf{Orientation:} The sign of \( \det(J) \) indicates whether the orientation is preserved or flipped.
\end{itemize}

\ex{Polar Coordinates}{
For the transformation \( T(r, \theta) = (r\cos\theta, r\sin\theta) \), the Jacobian matrix is:
\[
	J = \begin{bmatrix}
		\cos\theta & -r\sin\theta \\
		\sin\theta & r\cos\theta
	\end{bmatrix}
\]

The determinant is:
\[
	\det(J) = \begin{vmatrix}
		\cos\theta & -r\sin\theta \\
		\sin\theta & r\cos\theta
	\end{vmatrix}
	= r(\cos^2\theta + \sin^2\theta) = r
\]

Thus, the area differential in polar coordinates becomes:
\[
	dx \, dy = r \, dr \, d\theta
\]

}

\dfn{General Formula for Transforming Integrals}{
If \( T: S \rightarrow R \) is a transformation with Jacobian determinant \( |\det(J)| \), then the integral transforms as:
\[
	\iint_R f(x, y) \, dx \, dy = \iint_S f(T(u, v)) \, |\det(J)| \, du \, dv
\]

}

\dfn{Intuition for Higher Dimensions}{
In three dimensions, the Jacobian matrix extends to account for the transformation of volume elements:
\[
	J = \begin{bmatrix}
		\frac{\partial x}{\partial u} & \frac{\partial x}{\partial v} & \frac{\partial x}{\partial w} \\
		\frac{\partial y}{\partial u} & \frac{\partial y}{\partial v} & \frac{\partial y}{\partial w} \\
		\frac{\partial z}{\partial u} & \frac{\partial z}{\partial v} & \frac{\partial z}{\partial w}
	\end{bmatrix}
\]

The volume scaling factor is given by \( |\det(J)| \), and the integral transforms as:
\[
	\iiint_R f(x, y, z) \, dx \, dy \, dz = \iiint_S f(T(u, v, w)) \, |\det(J)| \, du \, dv \, dw
\]

}

\section{Non-overlapping from Mapping \( T \)}

\thm{Non-overlapping Condition}{
For any two points \( Q \) and \( P \):
\[
T(Q) \neq T(P) \quad \text{(This would result in overlapping areas in the domain \( R \))}
\]
However, boundaries (e.g., \( y = 2x \)) can overlap as long as the bounded region is distinct.
}

\ex{Integral Transformation Example}{
Evaluate:
\[
\iint_R 2x(y - 2x) \, dA
\]
where \( R \) is the parallelogram with vertices \((0, 0), (0, 1), (2, 4), (2, 3)\).

Steps:
\begin{enumerate}
    \item \textbf{Choose a Transformation:} Select a mapping \( T \) to simplify the integral.
    \item \textbf{Define the Mapping:}
    \[
    x = u, \quad y = 2x + v = 2u + v
    \]
    Substituting:
    \[
    (x, y) \rightarrow (u, v)
    \]

    \item \textbf{Boundary Equations:}
    \[
    0 \leq x \leq 2 \quad \Rightarrow \quad 0 \leq u \leq 2
    \]
    \[
    0 \leq y - 2x < 1 \quad \Rightarrow \quad 0 \leq v < 1
    \]

    \item \textbf{Region:}
    \[
    S = [0, 2] \times [0, 1]
    \]

    \item \textbf{Transform the Integrand:}
    \[
    f(T(x, y)) = 2u(v)
    \]

    \item \textbf{Jacobian Calculation:}
    \[
    J = \begin{bmatrix}
    x_u & x_v \\
    y_u & y_v
    \end{bmatrix}
    = \begin{bmatrix}
    1 & 0 \\
    2 & 1
    \end{bmatrix}, \quad \det(J) = 1 \cdot 1 - 2 \cdot 0 = 1
    \]

    \item \textbf{Transformed Integral:}
    \[
    \iint_R 2x(y - 2x) \, dA = \int_0^2 \int_0^1 2uv \, du \, dv
    \]
\end{enumerate}
}

\section{Integral Transformation for a Parallelogram Region}

\ex{Example of Transformation}{
Evaluate:
\[
\iint_R 2x(y - 2x) \, dA
\]
where \( R \) is the parallelogram defined by the vertices \((0, 0), (0, 1), (2, 4), (2, 3)\).

\subsection*{Steps:}
\begin{enumerate}
    \item \textbf{Choose a Transformation:} Select a transformation \( T \) that simplifies the integral.
    \item \textbf{Define \( x, y \) in terms of \( u, v \):}
    \[
    x = u, \quad y = 2x + v = 2u + v
    \]
    Substituting:
    \[
    (x, y) \rightarrow (u, v)
    \]
    Here, \( u \) corresponds to \( x \), and \( v = y - 2x \).

    \item \textbf{Boundary Equations:}
    \[
    0 \leq x \leq 2 \quad \Rightarrow \quad 0 \leq u \leq 2
    \]
    \[
    0 \leq y - 2x < 1 \quad \Rightarrow \quad 0 \leq v < 1
    \]

    \item \textbf{Region in \( u, v \):}
    \[
    S = [0, 2] \times [0, 1]
    \]
    This maps the parallelogram \( R \) into a rectangle \( S \) in the \( u, v \)-plane.

    \item \textbf{Transform the Integrand:}
    Substituting \( x = u \) and \( y - 2x = v \):
    \[
    f(T(x, y)) = 2u(v)
    \]

    \item \textbf{Jacobian Calculation:}
    The Jacobian matrix for the transformation \( T \) is:
    \[
    J = \begin{bmatrix}
    x_u & x_v \\
    y_u & y_v
    \end{bmatrix}
    = \begin{bmatrix}
    1 & 0 \\
    2 & 1
    \end{bmatrix}
    \]
    The determinant of \( J \) is:
    \[
    \det(J) = 1 \cdot 1 - 2 \cdot 0 = 1
    \]

    \item \textbf{Transformed Integral:}
    Using the transformation and the Jacobian determinant:
    \[
    \iint_R 2x(y - 2x) \, dA = \int_0^2 \int_0^1 2uv \, du \, dv
    \]
    The transformed integral simplifies the computation significantly.
\end{enumerate}
}

\section{Integral Transformation for a Triangular Region}

\ex{Example of Transformation}{
Evaluate:
\[
\iint_R (x - u)\sqrt{x - 2y} \, dA
\]
where \( R \) is the triangular region bounded by the lines \( y = 0 \), \( x - 2y = 0 \), and \( x = y + 1 \).

\subsection*{Steps:}
\begin{enumerate}
    \item \textbf{Region Definition:}
    The region \( R \) is defined by:
    \[
    y = 0, \quad x - 2y = 0, \quad x = y + 1
    \]
    The boundaries in \( x \) and \( y \) are:
    \[
    0 \leq x \leq 2, \quad 0 \leq y \leq \frac{x}{2}, \quad x \leq y + 1
    \]

    \item \textbf{Define Transformation:}
    Let:
    \[
    u = x - 2y, \quad v = x - y
    \]
    Substituting:
    \[
    x = v + u, \quad y = v - u
    \]

    \item \textbf{Boundaries in New Coordinates:}
    Using the transformation:
    \begin{align*}
    u &= x - 2y \quad \Rightarrow \quad 0 \leq u \leq 1 \\
    v &= x - y \quad \Rightarrow \quad u \leq v \leq 1
    \end{align*}

    The transformed region \( S \) is bounded by \( u = 0 \), \( v = 1 \), and \( v - u = 1 \).

    \item \textbf{Jacobian Calculation:}
    The Jacobian matrix for the transformation \( T(u, v) \) is:
    \[
    J = \begin{bmatrix}
    \frac{\partial x}{\partial u} & \frac{\partial x}{\partial v} \\
    \frac{\partial y}{\partial u} & \frac{\partial y}{\partial v}
    \end{bmatrix}
    = \begin{bmatrix}
    1 & 1 \\
    -2 & 1
    \end{bmatrix}
    \]
    The determinant of \( J \) is:
    \[
    \det(J) = (1)(1) - (1)(-2) = 1 + 2 = 3
    \]

    \item \textbf{Transform the Integral:}
    Using the transformation and Jacobian determinant:
    \[
    \iint_R (x - u)\sqrt{x - 2y} \, dA = \int_0^1 \int_0^v \sqrt{u} \cdot 3 \, du \, dv
    \]
    Simplify:
    \[
    \int_0^1 \int_0^v \sqrt{u} \, du \, dv = \int_0^1 \left[ \frac{2}{3} u^{3/2} \right]_0^v dv
    = \int_0^1 \frac{2}{3} v^{3/2} dv
    \]
    \[
    = \left[ \frac{2}{3} \cdot \frac{2}{5} v^{5/2} \right]_0^1 = \frac{4}{15}.
    \]
    The result is:
    \[
    \iint_R (x - u)\sqrt{x - 2y} \, dA = \frac{4}{15}.
    \]
\end{enumerate}
}


\end{document}