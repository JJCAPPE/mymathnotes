\documentclass{report}

\input{preamble}
\input{macros}
\input{letterfonts}

\usepackage{tikz}
\usepackage{tikz-3dplot}
\usepackage{amsmath}
\usepackage{amssymb}
\usepackage{pgfplots}
\usepackage{smartdiagram}
\usepackage{xcolor}
\usepackage{forest}
\usepackage{tikz-3dplot}
\usepgfplotslibrary{colormaps}
\usepgfplotslibrary{groupplots}
\pgfplotsset{compat=newest}

\usesmartdiagramlibrary{additions}

\title{\Huge{Temporary Doc}\\Calc 3}
\author{\huge{Giacomo Cappelletto}}
\date{23/10/24}

\begin{document}


\maketitle
\newpage
\pdfbookmark[section]{\contentsname}{toc}
\tableofcontents
\pagebreak

\chapter{Vector Valued Functions $f:\mathbb{R} \rightarrow \mathbb{R}^n$}

\ex{Problem with Multiple Surfaces}{
	The solid common to the cylinders bounded by \( z = \sin x \) and \( z = \sin y \) over the region
	\[
		R = \{ (x, y) : 0 \leq x \leq \pi, \; 0 \leq y \leq \pi \}.
	\]
	We define the domain \( D \) as
	\[
		D = \{ (x, y, z) : (x, y) \in R, \; 0 \leq z \leq \min(\sin x, \sin y) \}.
	\]

	To solve this problem, we will:

	\begin{enumerate}
		\item Consider one-fourth of the volume under the intersection of the two cylinders, taking advantage of symmetry.
		\item Identify the bounding conditions for \( z \) in each quadrant.
		\item Set up and compute the triple integral over \( D \) to find the volume.
	\end{enumerate}

	Given the symmetry, we restrict our analysis to the region
	\[
		0 \leq x \leq \frac{\pi}{2}, \quad 0 \leq y \leq \frac{\pi}{2}.
	\]
	In this quadrant, \( z \) is bounded by \( \min(\sin x, \sin y) \), which means \( z \leq \sin y \) in the region \( R_1 \), where \( \sin y \leq \sin x \).

	For region \( R_2 \) (where \( \sin x \leq \sin y \)), we have \( z \leq \sin x \). By observing symmetry, the volume in each quadrant contributes equally, so we can calculate the volume in this restricted region and multiply by 4.

	The bounds for \( x \) and \( y \) are:
	\[
		0 \leq y \leq \pi, \quad y \leq x \leq \pi - y.
	\]
	Thus, we set up the volume integral as follows:
	\[
		V = 4 \int_0^{\pi/2} \int_0^{\sin y} \int_y^{\pi - y} \, dx \, dy \, dz.
	\]

	Evaluating the integral:
	\[
		V = 4 \int_0^{\pi/2} \int_0^{\sin y} \left( \int_y^{\pi - y} dx \right) \, dy \, dz.
	\]

	Upon computation, this integral yields:
	\[
		V = \pi - 2.
	\]


	The key idea was to leverage the symmetry of the intersection region, focusing on one-fourth of the area and then scaling up by a factor of 4. By analyzing the geometry, we found that \( z \) was bounded by \( \sin y \) in region \( R_1 \). From there, the triple integral was computed over \( x \), \( y \), and \( z \) to yield the final volume of the region.
}

\begin{center}
	\begin{tikzpicture}
		\begin{axis}[
				width=12cm,
				height=12cm,
				view={70}{110},
				xlabel=$x$,
				ylabel=$y$,
				zlabel=$z$,
				xmin=0, xmax=3.1416,
				ymin=0, ymax=3.1416,
				zmin=0, zmax=1.2,
				grid=major,
				colormap/cool,
				xtick={0, 1.5708, 3.1416},
				xticklabels={$0$, $\frac{\pi}{2}$, $\pi$},
				ytick={0, 1.5708, 3.1416},
				yticklabels={$0$, $\frac{\pi}{2}$, $\pi$},
				samples=50,
				domain=0:3.1416,
				y domain=0:3.1416,
				trig format=rad,
			]

			% Surface for z = sin(x)
			\addplot3[
				surf,
				samples=50,
				domain=0:3.1416,
				y domain=0:3.1416,
				opacity=0.8,
			]
			{sin(x)};

			% Surface for z = sin(y)
			\addplot3[
				surf,
				samples=50,
				domain=0:3.1416,
				y domain=0:3.1416,
				opacity=0.8,
			]
			{sin(y)};

			% Boundary lines
			\addplot3[
				domain=0:3.1416,
				samples=50,
				thick,
				color=black,
			]
			({x},{0},{0});
			\addplot3[
				domain=0:3.1416,
				samples=50,
				thick,
				color=black,
			]
			({x},{3.1416},{0});
			\addplot3[
				domain=0:3.1416,
				samples=50,
				thick,
				color=black,
			]
			({0},{y},{0});
			\addplot3[
				domain=0:3.1416,
				samples=50,
				thick,
				color=black,
			]
			({3.1416},{y},{0});

			% Cross-section curves
			\addplot3[
				color=red,
				thick,
				domain=0:3.1416,
				samples=50,
			]
			({x},{1.5708},{sin(x)});

			\addplot3[
				color=blue,
				thick,
				domain=0:3.1416,
				samples=50,
			]
			({1.5708},{y},{sin(y)});

			% Height indicator lines
			\foreach \x in {0.5,1,1.5} {
					\addplot3[
						dashed,
						color=gray,
					] coordinates {
							(0,\x,0)
							(3.1416,\x,0)
						};
					\addplot3[
						dashed,
						color=gray,
					] coordinates {
							(\x,0,0)
							(\x,3.1416,0)
						};
				}

			% Add legend
			\addlegendentry{$z = \sin(x)$}
			\addlegendentry{$z = \sin(y)$}
			\addlegendentry{Boundaries}
			\addlegendentry{$x$-cross section}
			\addlegendentry{$y$-cross section}

		\end{axis}
	\end{tikzpicture}
\end{center}

\end{document}