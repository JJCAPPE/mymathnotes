\documentclass{report}

\input{preamble}
\input{macros}
\input{letterfonts}

\usepackage{tikz}
\usepackage{tikz-3dplot}
\usepackage{amsmath}
\usepackage{pgfplots}
\usepackage{smartdiagram}
\usesmartdiagramlibrary{additions}
\usepackage{xcolor}
\usepackage{forest}
\usepgfplotslibrary{colormaps}
\usepgfplotslibrary{groupplots}
\usepgfplotslibrary{polar}
\pgfplotsset{compat=newest}
\tikzset{>=latex}
\usepackage{siunitx}

\title{\Huge{MA193}\\Discrete Mathematics}
\author{\huge{Giacomo Cappelletto}}
\date{21/1/25}

\begin{document}


\maketitle
\newpage
\pdfbookmark[section]{\contentsname}{toc}
\tableofcontents
\pagebreak

\chapter{Fundamental Principles of Counting}

\section{Counting with Repetitions}

\nt{
	These notes cover the basic counting principles (often called the
	``Rule of Sum'' and the ``Rule of Product''), along with a brief
	discussion of permutations and combinations, including the formula
	for permutations of multiset objects (i.e., repeated elements).
}

\dfn{Rule of Sum (``OR'')}{
	If a certain task can be done in \(n\) ways and another independent
	task can be done in \(m\) ways, and these tasks are mutually exclusive,
	then there are \(n + m\) ways to do \textit{one} of the two tasks.
}

\dfn{Rule of Product (``AND'')}{
	If a procedure can be broken into two consecutive steps such that
	the first step can be done in \(n\) ways and the second step can be
	done in \(m\) ways (independently of how the first step is done),
	then there are \(n \times m\) ways to do the entire procedure.
}

\nt{
	In many counting problems, we break a larger procedure into a series
	of smaller steps and then apply either the Rule of Sum or the Rule
	of Product (or both) as needed.
}

\dfn{Arrangements of \(n\) Distinct Objects}{
	If you want to arrange \(n\) distinct objects in a row (i.e., an
	ordered list), there are \(n!\) ways to do so. Order matters here,
	and this number is referred to as the number of permutations of
	\(n\) distinct items.
}

\dfn{Permutations of Multisets}{
	Suppose we have \(n\) total objects, but they are not all distinct.
	Instead, let there be \(n_1\) objects of type 1, \(n_2\) objects of
	type 2, \(\dots\), and \(n_k\) objects of type \(k\). Clearly
	\[
		n_1 + n_2 + \cdots + n_k = n.
	\]
	Then the number of distinct ways to arrange all \(n\) objects is
	\[
		\frac{n!}{n_1! \, n_2! \, \dots \, n_k!}.
	\]
}

\ex{Examples of Counting with Repetitions}{
	\begin{enumerate}
		\item \textbf{ABCD:} All letters are distinct, so the number of ways
		      to arrange A, B, C, D is
		      \[
			      4! = 24.
		      \]
		\item \textbf{AABC:} Here we have 4 total letters, with A repeated twice.
		      The number of distinct arrangements is
		      \[
			      \frac{4!}{2!} = \frac{24}{2} = 12.
		      \]
		\item \textbf{AABB:} Now we have 2 A's and 2 B's (4 letters total). The
		      number of arrangements is
		      \[
			      \frac{4!}{2! \, 2!} = \frac{24}{2 \times 2} = 6.
		      \]
		\item \textbf{SUCCESS:} The word ``SUCCESS'' has 7 letters total:
		      3 S's, 2 C's, 1 U, and 1 E. The number of distinct permutations is
		      \[
			      \frac{7!}{3! \, 2! \, 1! \, 1!} = \frac{5040}{(6)(2)(1)(1)} = 420.
		      \]
	\end{enumerate}
}

\section*{Committee-Choosing Problems}

\dfn{Combinations}{
	A \emph{combination} is a way to choose \(r\) objects from \(n\) distinct
	objects where order does \emph{not} matter. The number of ways to do
	so is given by
	\[
		\binom{n}{r} \;=\; \frac{n!}{r! \,(n-r)!}.
	\]
	In many committee-selection problems, we use combinations because the
	particular order in which people are chosen does not matter.
}

\dfn{When to Use Permutations vs.\ Combinations}{
	\begin{itemize}
		\item \emph{Permutations} (\(P(n,r)\) or \(\frac{n!}{(n-r)!}\)) are used
		      when we care about the \emph{order} of the chosen items (for instance,
		      arranging people in a line for a photo).
		\item \emph{Combinations} (\(\binom{n}{r}\)) are used when we only care
		      about \emph{which} items are chosen, not the order in which they appear
		      (e.g., forming committees).
	\end{itemize}
}

\dfn{Basic Subset Counting}{
	Note that the total number of distinct subsets of a set with \(n\)
	elements is \(2^n\). This comes from the fact that, for each element,
	we independently choose to include it or not in a given subset.
	For an \(r\)-element subset, specifically, we use \(\binom{n}{r}\).
}

\ex{Choosing a Simple Committee}{
	Suppose we have 8 people in a group, and we want to choose a committee
	of 3 individuals. Since order does not matter, the number of ways to
	choose the committee is
	\[
		\binom{8}{3} \;=\; \frac{8!}{3!(8-3)!} \;=\; \frac{8!}{3!5!} \;=\; 56.
	\]
}

\ex{Committee with Restrictions: Gender Balance}{
	Imagine we have 5 men and 6 women, and we want to form a committee of 4
	people that has at least 2 women. We can break it down by the number
	of women selected:
	\[
		\binom{6}{2}\binom{5}{2} \;+\;
		\binom{6}{3}\binom{5}{1} \;+\;
		\binom{6}{4}\binom{5}{0}.
	\]
	Here, \(\binom{6}{k}\) chooses the women, and \(\binom{5}{4-k}\) chooses
	the men for each scenario. Summing these counts gives the total number
	of ways to form such a committee.
}
\dfn{Labeled Teams}{%
	When teams themselves have distinct “names” or labels (e.g.\ Team~A vs.\ Team~B).
	A configuration where Person~1 is on Team~A and Person~2 is on Team~B
	is then different from Person~1 on Team~B and Person~2 on Team~A.}

\dfn{Unlabeled Teams}{%
	When the teams are treated as identical (no distinct labels).
	In that scenario, swapping membership between “Team~A” and “Team~B”
	would not create a new configuration.}
\ex{Committee with Subgroups Required}{
	Say we have 10 people, of whom 3 are Teaching Assistants (TAs) and 7 are
	Professors, and we wish to form a committee of 4 people \emph{with exactly
		2 TAs}. We select 2 from the 3 TAs and 2 from the 7 Professors, yielding
	\[
		\binom{3}{2} \;\times\; \binom{7}{2}
	\]
	possible committees.
	\nt{%
		\textbf{Labeled vs.\ Unlabeled Teams.}
		``3 people, 2 teams with a label?'' Here, we wonder if putting
		Alice on Team~A and Bob on Team~B differs from Alice on Team~B
		and Bob on Team~A. If teams are unlabeled, then swapping them
		would not create a different outcome. If teams \emph{are} labeled,
		we count each distinct assignment as different.
	}
}

\ex{Larger Committees: Multiple Constraints}{
	Suppose there are 12 people divided into three categories: 4 from group
	A, 5 from group B, and 3 from group C. We want a committee of 5 that
	has at least 1 person from each group. One way to count is to enumerate
	possible splits of 5 among \((A,B,C)\), ensuring each category has at
	least one representative. For instance:
	\begin{itemize}
		\item 1 from A, 3 from B, 1 from C,
		\item 1 from A, 2 from B, 2 from C,
		\item 2 from A, 2 from B, 1 from C,
		\item and so forth,
	\end{itemize}
	and sum the corresponding products of binomial coefficients
	\(\binom{4}{\dots}\binom{5}{\dots}\binom{3}{\dots}\).
}

\nt{
	These examples illustrate the typical approaches to \emph{committee
		choosing} problems:
	\begin{itemize}
		\item Identify if order matters (usually it does not, hence combinations).
		\item If there are constraints (e.g.\ minimum number of members from a certain
		      group), split the problem into valid cases and sum their respective counts.
	\end{itemize}
}

\section{Gambling}
\ex{Probability of Getting a Flush in Poker}{
	A \emph{flush} in poker is a hand where all 5 cards are of the same suit.
	To calculate the probability of being dealt a flush, we proceed as follows:

	\begin{itemize}
		\item There are 4 suits in a deck (hearts, diamonds, clubs, spades).
		\item For each suit, there are \(\binom{13}{5}\) ways to choose 5 cards from the 13 available.
		\item Thus, the total number of ways to get a flush is \(4 \times \binom{13}{5}\).
		\item The total number of 5-card hands from a 52-card deck is \(\binom{52}{5}\).
	\end{itemize}

	Therefore, the probability \(P\) of being dealt a flush is given by
	\[
		P(\text{flush}) = \frac{\binom{4}{1} \times \binom{13}{5}}{\binom{52}{5}}.
	\]

	So, the probability of being dealt a flush in poker is approximately 0.198\%.
}




\ex{Example: 3 people into 2 teams}{%
	Suppose we have 3 people (say Alice, Bob, and Carol).
	If the teams are \emph{labeled}, we count every distinct assignment
	into Team~A vs.\ Team~B.
	If the teams are \emph{unlabeled}, we only care about who ends up together,
	not which group is called “Team~A.”
	Hence the total counts differ, because labeling typically doubles
	the number of distinct ways (unless one team is empty).}

\nt Next, consider two common 5-card poker hands:

\dfn{One Pair}{%
	A 5-card hand containing exactly two cards of the same rank
	and the other three cards all of different ranks (and each different
	from the pair’s rank).}

\ex{Counting One Pair}{%
	To choose exactly one pair out of a standard deck, we do:
	\[
		\binom{13}{1}\binom{4}{2}
	\]
	to pick which rank we have a pair of (13 ways)
	and which 2 suits out of the 4.
	For a complete 5-card hand with “exactly one pair,”
	we then choose the other 3 cards of distinct ranks,
	leading to the well-known formula
	\(
	\binom{13}{1}\binom{4}{2}\,\times\,\binom{12}{3}\,(\text{each chosen rank has } \binom{4}{1} \text{ ways}).
	\)
}

\dfn{Full House}{%
	A 5-card hand consisting of 3 cards of one rank plus 2 cards of another rank.}

\ex{Counting a Full House}{%
	We choose the rank for the three-of-a-kind in \(\binom{13}{1}\) ways,
	and then choose 3 suits out of 4 for that rank: \(\binom{4}{3}\).
	Next, we choose the rank for the pair out of the remaining 12 ranks:
	\(\binom{12}{1}\), and choose 2 suits out of 4 for that pair: \(\binom{4}{2}\).
	Hence,
	\[
		\text{Number of Full Houses}
		\;=\; \binom{13}{1}\binom{4}{3}\,\times\,\binom{12}{1}\binom{4}{2}.
	\]
}

\section{Binomial identities}

\dfn{Binomial Symmetry}{%
	\centering
	\(\displaystyle \binom{n}{k} \;=\; \binom{n}{n-k}.\)
}
\nt{The intuition is that choosing \(k\) objects out of \(n\)
	is equivalent to choosing which \(n-k\) to \emph{leave out}.
	Hence \(\binom{n}{k}\) equals \(\binom{n}{n-k}\).
}
\ex{Example}{%
	For instance, \(\binom{5}{2} = \binom{5}{3}\).
	Choosing 2 items from 5 is the same as deciding which 3 are excluded.
}

\dfn{Pascal’s Rule}{%
	\centering
	\(\displaystyle \binom{n}{v} \;=\; \binom{n-1}{v-1} + \binom{n-1}{v}.\)
}
\nt{A combinatorial interpretation is to imagine a set of \(n\) items
	where one particular “special” item can either be in your chosen subset or not.
	If you include the special item, then you need to pick the remaining \(v-1\) items
	from the other \(n-1\).
	If you exclude the special item, then you pick all \(v\) items from the other \(n-1\).
	Summing these counts gives \(\binom{n}{v}\).
}
\ex{Example}{%
	\(\binom{5}{3} = \binom{4}{2} + \binom{4}{3}.\)
	Either include a special item (then choose 2 more from the other 4)
	or exclude it (then choose all 3 from the other 4).
}

\dfn{Hockey-Stick Identity}{%
	\centering
	\(\displaystyle
	\sum_{j=r}^n \binom{j}{r}
	\;=\; \binom{n+1}{r+1}.
	\)
}
\nt{This identity is sometimes called the “Christmas Stocking” or “Hockey-Stick” identity
	because of the pattern it creates in Pascal’s triangle.
	It can be viewed as a cumulative version of Pascal’s Rule:
	once you fix \(\binom{n+1}{r+1}\), you can “unroll” it into the sum
	\(\binom{r}{r} + \binom{r+1}{r} + \dots + \binom{n}{r}\).
}
\ex{Example}{%
	\(\binom{3}{3} + \binom{4}{3} + \binom{5}{3} \;=\; \binom{6}{4}.\)
	In general, if you sum along a diagonal in Pascal’s triangle,
	you end up at a binomial coefficient one row down and one column over.
}


\nt{
	A useful application of these identities is to evaluate sums of products
	like \(1 \cdot 2 \cdot 3 + 2 \cdot 3 \cdot 4 + \dots + (u-2)(u-1)u\).
	Note that
	\[
		k(k+1)(k+2) \;=\; 3! \,\binom{k+2}{3},
	\]
	so
	\[
		\sum_{k=1}^{u-2} k(k+1)(k+2)
		\;=\; 3!\,\sum_{k=1}^{u-2} \binom{k+2}{3}
		\;=\; 3! \,\binom{u+1}{4}.
	\]
}
\ex{Binomial Expansion}{%
\(\displaystyle (x + y)^n
\;=\; \sum_{k=0}^n \binom{n}{k}\,x^{n-k}\,y^k.\)
For instance, \((2x + 3y)^9\) expands as
\[
	\sum_{k=0}^9 \binom{9}{k} (2x)^{9-k} (3y)^k.
\]
}

\nt{
	Recall the binomial expansion:
	\[
		(1 + x)^n \;=\; \binom{n}{0}
		\;+\; \binom{n}{1}x
		\;+\; \binom{n}{2}x^2
		\;+\;\dots\;+\; \binom{n}{n}x^n.
	\]
	If we substitute \(x = -2\), we obtain
	\[
		\sum_{k=0}^{n} \binom{n}{k}(-2)^k.
	\]
	This is simply an application of the binomial theorem with a negative value for \(x\).
}

\dfn{Sum of Squares}{
	A well-known formula for the sum of the first \(n\) squares is
	\[
		1^2 + 2^2 + 3^2 + \dots + n^2 \;=\;\sum_{k=1}^n k^2 \;=\; \frac{n(n+1)(2n+1)}{6}.
	\]
	One way to derive or interpret this is via Riemann sums, though it can also be shown using induction or other combinatorial arguments.
}

\nt{
	We can also consider the derivative approach related to \((1 + x)^n\). Note that
	\[
		\frac{d}{dx}(1 + x)^n \;=\; n (1 + x)^{n-1}.
	\]
	Expanding \((1 + x)^n\) term-by-term:
	\[
		(1 + x)^n
		\;=\; \binom{n}{0} + \binom{n}{1} x + \binom{n}{2} x^2 + \cdots + \binom{n}{n} x^n,
	\]
	so taking a derivative and then comparing coefficients often provides a way to form identities involving sums of binomial coefficients.
}

\dfn{Multinomial Coefficients}{
	For nonnegative integers \(n\) and nonnegative integers \(k_1, k_2, \dots, k_m\) such that \(k_1 + k_2 + \cdots + k_m = n\), the multinomial coefficient is defined as
	\[
		\binom{n}{k_1, k_2, \dots, k_m}
		\;=\;
		\frac{n!}{k_1! \, k_2! \cdots k_m!}.
	\]
	This appears when expanding expressions of the form
	\[
		(x_1 + x_2 + \cdots + x_m)^n.
	\]
}

\ex{Expansion of \((a+b+c)^4\)}{
	Consider the expansion
	\[
		(a + b + c)^4.
	\]
	If we want, for example, the coefficient of \(b^2\,c^2\) in this expansion, we look at
	\[
		\binom{4}{0,2,2} \;=\; \frac{4!}{0! \, 2! \, 2!} \;=\; 6.
	\]
	So the term corresponding to \(b^2c^2\) in \((a+b+c)^4\) is \(6\,b^2c^2\).
}

\ex{Expansion of \((\,a + 2b + 3c - 2d + 5)^{16}\)}{
	As a more general example, for the expansion of
	\[
		(a + 2b + 3c - 2d + 5)^{16},
	\]
	one can seek a specific term’s coefficient, say the coefficient of \(a^2 b^3 c^1 d^5\). One would use the multinomial theorem:
	\[
		\binom{16}{2,3,1,5,\,16-(2+3+1+5)}(a)^2 (2b)^3 (3c)^{1} (-2d)^{5}(5)^{\,16-(2+3+1+5)}.
	\]
	Every factor (such as \(2b\), \(3c\), or \(-2d\)) contributes appropriately to the power and to the overall coefficient.
}

\nt{
	A classic combinatorial problem is \emph{dealing cards in a Bridge game}. A standard deck has \(52\) cards, dealt evenly to \(4\) players (each player receives \(13\) cards). For instance:
	\[
		\binom{52}{13,13,13,13}
	\]
	is the total number of ways to deal the entire deck to \(4\) players.
}

\ex{Probability that two specific players get all the spades}{
	There are \(13\) spades in the deck. If we ask for the probability that two given players split all \(13\) spades between them (i.e., they collectively get all spades), we count how the \(13\) spades can be distributed to those two players, and how the remaining \(39\) cards are distributed among all four players. Such problems are typically approached using multinomial coefficients or combinations, depending on how strictly the spades must be divided.
}

\nt{
	Another recurring theme is \emph{distributing indistinguishable objects} into distinct boxes. For example, distributing \(20\) donuts (indistinguishable) among \(r\) different people or “flavors.” The number of ways to do this (allowing any number of donuts per person) is given by:
	\[
		\binom{20 + r - 1}{r - 1}
	\]
	in the classical “stars and bars” argument. More generally, the number of nonnegative integer solutions to
	\[
		A + B + \cdots + (\,\text{n variables}) \;=\; N
	\]
	is
	\[
		\binom{N + \text{n} - 1}{\text{n} - 1}.
	\]
}

\nt{
\textbf{Placing Distinct vs.~Indistinguishable Objects into Containers}

\begin{itemize}
	\item \emph{Distinct objects into distinct containers:} If there are \(r\) distinct objects and \(n\) distinct containers, each object can go into any of the \(n\) containers. Hence there are 
	\[
	n^r
	\]
	ways to place them.
	\item \emph{Distinct objects into identical containers:} This is related to set partitions. We ask: in how many ways can a set of \(r\) distinct elements be partitioned into (up to) \(n\) unlabeled subsets? Such counting typically involves the Stirling numbers of the second kind, denoted \(\displaystyle S(r,n)\). 
	\item \emph{Indistinguishable objects into distinct containers:} This is a “stars and bars” problem. If \(r\) identical objects are to be placed in \(n\) distinct containers, with no restriction on how many objects go in each container (i.e.\ allowing zero), the number of ways is
	\[
	\binom{r + n - 1}{n - 1}.
	\]
	\item \emph{Indistinguishable objects into identical containers:} This typically counts the number of partitions of an integer \(r\) into at most \(n\) parts. Equivalently, it’s the number of ways to write
	\[
	r = x_1 + x_2 + \cdots + x_n
	\]
	where \(x_i \ge 0\) and the order of parts does not matter. These numbers connect to the theory of integer partitions, which can be considerably more involved.
\end{itemize}
}

\ex{Two Distinct Objects in Three Identical Containers}{
We have two distinct objects (call them \(A\) and \(B\)) and three identical “boxes” (unlabeled). One way to count the distributions is to look at all possible groupings:
\begin{itemize}
	\item Both \(A\) and \(B\) in the same container (there are 3 ways if the containers were distinct, but only 1 way when the containers are identical).
	\item \(A\) and \(B\) in different containers (again, if containers were distinct, we’d count more, but with identical containers we see that splitting them up is effectively one arrangement).
\end{itemize}
Hence there are 2 possible ways:
\[
\{A,B\}\,,\quad \{A\},\{B\}.
\]
}

\ex{Distribution of Toys and Candy}{
Suppose we have 7 identical toys and 8 identical candy bars, and 4 kids. Each kid must receive exactly 1 toy (so that accounts for 4 toys). Then 3 toys remain to be distributed freely among the 4 kids. The number of ways to distribute these 3 identical toys into 4 distinct kids is
\[
\binom{3 + 4 - 1}{4 - 1} \;=\; \binom{6}{3} \;=\; 20.
\]
Similarly, all 8 candies are to be distributed among 4 kids. If there is no restriction (i.e.\ any kid can get any number of candies), the count is
\[
\binom{8 + 4 - 1}{4 - 1} \;=\; \binom{11}{3}.
\]
The overall number of ways to distribute both sets of items is then the product of these two binomial coefficients (because the toy distribution and candy distribution choices are independent).
}

\dfn{Compositions of an Integer}{
A \emph{composition} of a positive integer \(n\) is a way of writing \(n\) as an ordered sum of positive integers. For instance, the integer \(4\) has the following compositions:
\[
4, \quad 3+1,\quad 1+3,\quad 2+2,\quad 2+1+1,\quad 1+2+1,\quad 1+1+2,\quad 1+1+1+1.
\]
If one writes \(n\) as \(x_1 + x_2 + \dots + x_k\) with each \(x_i \ge 1\), the order matters for compositions. The total number of compositions of \(n\) is \(2^{\,n-1}\).

\begin{itemize}
	\item If we allow zeros as parts, i.e.\ \(x_i \ge 0\), we get related formulas often solved by “stars and bars.”
	\item Partitions, on the other hand, disregard order; they look at the sum of positive parts in a non-increasing arrangement, etc.
\end{itemize}
}

\ex{Counting Runs in a Sequence of Coin Flips}{
Consider flipping a fair coin 15 times and recording a sequence of heads (H) and tails (T). A \emph{run} is a maximal consecutive string of identical outcomes: for instance, in the sequence
\[
H\; H\; T\; T\; T\; H\; H \;(\dots)
\]
we have runs \(\{HH\}, \{TTT\}, \{HH\}, \dots\). One might ask for the probability that exactly 7 runs occur. Such problems are typically handled by combinatorial arguments or recursion, counting all sequences that produce exactly that number of runs. 
}



\end{document}