\documentclass{report}

\input{preamble}
\input{macros}
\input{letterfonts}

\usepackage{tikz}
\usepackage{tikz-3dplot}
\usepackage{amsmath}
\usepackage{pgfplots}
\usepackage{smartdiagram}
\usesmartdiagramlibrary{additions}
\usepackage{xcolor}
\usepackage{forest}
\usepgfplotslibrary{colormaps}
\usepgfplotslibrary{groupplots}
\usepgfplotslibrary{polar}
\pgfplotsset{compat=newest}
\tikzset{>=latex}
\usepackage{siunitx}

\title{\Huge{MA193}\\Discrete Mathematics}
\author{\huge{Giacomo Cappelletto}}
\date{21/1/25}

\begin{document}


\maketitle
\newpage
\pdfbookmark[section]{\contentsname}{toc}
\tableofcontents
\pagebreak

\chapter{Fundamental Principles of Counting}

\section{Counting with Repetitions}

\nt{
These notes cover the basic counting principles (often called the
``Rule of Sum'' and the ``Rule of Product''), along with a brief
discussion of permutations and combinations, including the formula
for permutations of multiset objects (i.e., repeated elements).
}

\dfn{Rule of Sum (``OR'')}{
If a certain task can be done in \(n\) ways and another independent
task can be done in \(m\) ways, and these tasks are mutually exclusive,
then there are \(n + m\) ways to do \textit{one} of the two tasks.
}

\dfn{Rule of Product (``AND'')}{
If a procedure can be broken into two consecutive steps such that
the first step can be done in \(n\) ways and the second step can be
done in \(m\) ways (independently of how the first step is done),
then there are \(n \times m\) ways to do the entire procedure.
}

\nt{
In many counting problems, we break a larger procedure into a series
of smaller steps and then apply either the Rule of Sum or the Rule
of Product (or both) as needed.
}

\dfn{Arrangements of \(n\) Distinct Objects}{
If you want to arrange \(n\) distinct objects in a row (i.e., an
ordered list), there are \(n!\) ways to do so. Order matters here,
and this number is referred to as the number of permutations of
\(n\) distinct items.
}

\dfn{Permutations of Multisets}{
Suppose we have \(n\) total objects, but they are not all distinct.
Instead, let there be \(n_1\) objects of type 1, \(n_2\) objects of
type 2, \(\dots\), and \(n_k\) objects of type \(k\). Clearly
\[
n_1 + n_2 + \cdots + n_k = n.
\]
Then the number of distinct ways to arrange all \(n\) objects is
\[
\frac{n!}{n_1! \, n_2! \, \dots \, n_k!}.
\]
}

\ex{Examples of Counting with Repetitions}{
\begin{enumerate}
\item \textbf{ABCD:} All letters are distinct, so the number of ways
to arrange A, B, C, D is
\[
4! = 24.
\]
\item \textbf{AABC:} Here we have 4 total letters, with A repeated twice.
The number of distinct arrangements is
\[
\frac{4!}{2!} = \frac{24}{2} = 12.
\]
\item \textbf{AABB:} Now we have 2 A's and 2 B's (4 letters total). The
number of arrangements is
\[
\frac{4!}{2! \, 2!} = \frac{24}{2 \times 2} = 6.
\]
\item \textbf{SUCCESS:} The word ``SUCCESS'' has 7 letters total:
3 S's, 2 C's, 1 U, and 1 E. The number of distinct permutations is
\[
\frac{7!}{3! \, 2! \, 1! \, 1!} = \frac{5040}{(6)(2)(1)(1)} = 420.
\]
\end{enumerate}
}

\end{document}